%%%%%%%%%%%%%%%%%
% This is an sample CV template created using altacv.cls
% (v1.1.2, 1 February 2017) written by LianTze Lim (liantze@gmail.com). Now compiles with pdfLaTeX, XeLaTeX and LuaLaTeX.
%
%% It may be distributed and/or modified under the
%% conditions of the LaTeX Project Public License, either version 1.3
%% of this license or (at your option) any later version.
%% The latest version of this license is in
%%    http://www.latex-project.org/lppl.txt
%% and version 1.3 or later is part of all distributions of LaTeX
%% version 2003/12/01 or later.
%%%%%%%%%%%%%%%%

%% If you need to pass whatever options to xcolor
\PassOptionsToPackage{dvipsnames}{xcolor}

%% If you are using \orcid or academicons
%% icons, make sure you have the academicons
%% option here, and compile with XeLaTeX
%% or LuaLaTeX.
% \documentclass[10pt,a4paper,academicons]{altacv}

%% Use the "normalphoto" option if you want a normal photo instead of cropped to a circle
% \documentclass[10pt,a4paper,normalphoto]{altacv}

\documentclass[8pt,a4paper]{altacv}

%% AltaCV uses the fontawesome and academicon fonts
%% and packages.
%% See texdoc.net/pkg/fontawecome and http://texdoc.net/pkg/academicons for full list of symbols.
%%
%% Compile with LuaLaTeX for best results. If you
%% want to use XeLaTeX, you may need to install
%% Academicons.ttf in your operating system's font
%% folder.


% Change the page layout if you need to
\geometry{left=1cm,right=9cm,marginparwidth=6.8cm,marginparsep=1.2cm,top=0.5cm,bottom=1.25cm}

% Change the font if you want to.

% If using pdflatex:
\usepackage[utf8]{inputenc}
\usepackage[T1]{fontenc}
\usepackage[default]{lato}
\usepackage{hyperref}
\usepackage[document]{ragged2e}
% If using xelatex or lualatex:
% \setmainfont{Lato}

% Change the colours if you want to
\definecolor{Mulberry}{HTML}{72243D}
\definecolor{SlateGrey}{HTML}{2E2E2E}
\definecolor{LightGrey}{HTML}{666666}
\colorlet{heading}{Sepia}
\colorlet{accent}{Mulberry}
\colorlet{emphasis}{SlateGrey}
\colorlet{body}{LightGrey}

% Change the bullets for itemize and rating marker
% for \cvskill if you want to
\renewcommand{\itemmarker}{{\small\textbullet}}
\renewcommand{\ratingmarker}{\faCircle}

% Added by maxence 
\newcommand{\cvsubevent}[4]{%
\medskip
\ifstrequal{#1}{}{}{
\textbf{\color{emphasis}#1 -- \color{accent}#2}}\par
  \ifstrequal{#3}{}{}{\small\makebox[0.5\linewidth][l]{\faCalendar~#3}}\par
  \ifstrequal{#4}{}{}{}
  \smallskip\normalsize

}


\newcommand{\cveventtechnologies}[1]{%
  \smallskip\normalsize
  \ifstrequal{#1}{}{}{\small\makebox[0.5\linewidth][l]{\faCode~#1}}
  \smallskip\normalsize
}


\begin{document}

\fr{\name{Maxence Labusquière}}
\eng{\name{Maxence Labusquiere}}
\tagline{
\fr{Ingénieur développeur}
\eng{Senior software engineer}
}


\fr{\photo{3.2cm}{mlabusquiere}}
\personalinfo{%
  % Not all of these are required!
  % You can add your own with \printinfo{symbol}{detail}
  \email{labusquiere@gmail.com}
  \phone{+31 (0)6 51 40 61 28}
  \mailaddress{Kattenburgerhof 48, 1018 KD Amsterdam}
  %% Keep this jump
  
  \smallskip
  \twitter{@mlabusquiere}
  \github{github.com/mlabusquiere}
  %% You MUST add the academicons option to \documentclass, then compile with LuaLaTeX or XeLaTeX, if you want to use \orcid or other academicons commands.
%   \orcid{orcid.org/0000-0000-0000-0000}
}

%% Make the header extend all the way to the right, if you want.
\begin{fullwidth}
\makecvheader
\end{fullwidth}



% Definitely position date smaller. Main date a bit bigger?



%% Provide the file name containing the sidebar contents as an optional parameter to \cvsection.
%% You can always just use \marginpar{...} if you do
%% not need to align the top of the contents to any
%% \cvsection title in the "main" bar.
\cvsection[mlabusquiere_p1_sidebar]{\fr{Expérience}\eng{Experience}}

\fr{\cvevent{Picnic - Supermarché en ligne}{Tech Lead / Backen Software engineer}{Septembre 2019 -- Aujourd'hui}{Amsterdam - Pays-bas}}
\eng{\cvevent{Picnic - Online Supermarket}{Tech Lead / Backend Software engineer}{September 2019 -- Today}{Amsterdam - Netherlands}}
\fr{Participation active et evolution au sein au scale up d'un tech department multi-culturel de 40 a 300 pers
La flexibilite}
\eng{During my 4-year tenure in a fast-paced, multicultural scale-up environment, I embraced multiple roles and diversified projects, enabling me to continuously enhance my soft and technical skills and contribute significantly }
\smallskip

\begin{itemize} 
\fr{ 
  \item todo 
}
\eng{
  \item Lead and scale an agile team from 3 backends to 7 multi-disciplinary team members
  \item Responsible for HR performance reviews and fostering growth through coaching    
  \item Participate in roadmap processes and responsible for team deliveries
  \item Contribute and improve requirement gathering, design process and API definition 
  \item Migrate monoliths to a modularized microservice architecture
  \item Weekly PR contributor on 20+ projects mostly in Java 
} 
\end{itemize}
\cveventtechnologies{Java 21 - SpringBoot - Reactor - MongoDB - RabbitMQ - Kubernetes - Elasticsearch - BDD}

\fr{\cvsubevent{Développeur}{Warehouse - WS Orchestration}{Décembre 2022 -- Aujourd'hui}{}}
\eng{\cvsubevent{Software Engineer}{Warehouse - WS Orchestration }{December 2022 - Today}{}}
\fr{Todo}
\eng{Orchestration is responsible for different services aiming to optimize the tote movements in the newly modularized warehouse system 
and of the inventory management}

\fr{\cvsubevent{Tech lead}{Consumer - Merchandising}{Mai 2021 -- Octobre 2022}}
\eng{\cvsubevent{Tech lead}{Consumer - Merchandising}{May 2021 -- October 2022}}
\fr{todo}
\eng{Merchandising is responsible for all configuration tools and associated backends for the search, the discount and personalization of the Picnic online application}

\fr{\cvsubevent{Développeur}{Store}{Mai 2019 -- Octobre 2022}}
\eng{\cvsubevent{Software Engineer}{Store}{September 2019 -- May 2021}}
\fr{todo}
\eng{Store is responsible for the customer mobile application}

\bigskip
\fr{\cvevent{Zenika - Pôle IKI Consulting}{Consultant / Formateur}{Septembre 2014 -- September 2019}{Paris - France}}
\eng{\cvevent{Zenika - Consulting}{Consultant / Trainer}{September 2014 -- September 2019}{Paris - France}}

\fr{Implication dans le development de l'agence :}
\eng{During 5 years as part of a scaling architecture and agile consultancy firm, I 
supported many corporate customers in their respective challenges by adopting various roles and proning technical excellence.
Through these diversified experiences, I gathered a solid knowledge of Java and its ecosystem, migrating applications from monoliths to event-driven microservice architecture and software engineering methodologies. I also contributed into the agency's development:
}
\smallskip
\begin{itemize}
\fr{
  \item Animation d'une trentaine de séances formations (Spring Integration, RabbitMQ, Microservices, Maven, Git, Java)
  \item Réalisation d'une douzaine de missions d'expertises courtes (RabbitMQ, Java, Jenkins)
  \item Contribution aux appels d'offre
  \item Participation au recrutement des consultants
  \item Animation d'événements pour la tribu Architecture
}

\eng{
  \item Gave over 30 training sessions and 12 short expert missions
  \item Contributed to recruitment and acquisition
  \item Organized several internal architecture events
}
\end{itemize}

% Add JC decaux

% Detaile que les missions ont tous en commun un rapport a la production et la modification d'un existant qui fonctione

\fr{\cvsubevent{Tech lead}{Altice (SFR) - PFS TV - Sekai}{Mars 2017 --  Janvier 2019}{Saint Denis - France}}
\eng{\cvsubevent{Tech lead}{Altice (SFR) - PFS TV - Sekai}{March 2017 -- January 2019}{Saint Denis - France}}

\fr{Sekai est l'équipe s'occupant de la refonte des backends de setup des objets IoT SFR et de toutes les applications autour de la TV}
\eng{Sekai is the team that overhaul all backends concerning the tv setup}

\fr{Responsable de la nouvelle application parametrant les acces utilisateurs et du Microservice chassi framework}
\eng{Responsible for the new application allowing user access parametrization and the Microservice chassis framework}

\cveventtechnologies{Java - SpringBoot - Couchbase - Kafka - Docker - Swarm - Cucumber}

\fr{\cvsubevent{Tech lead}{Kuka Robotics - Gaia}{Octobre 2016 -- Mars 2017}{Massy-Palaiseau - France}}
\eng{\cvsubevent{Tech lead}{Kuka Robotics - Gaia}{October 2016 -- March 2017}{Massy-Palaiseau - France}}
% Ameliorer FR
\fr{Gaia est un framework permettant d'adapter sans développeurs les processus métiers du robot IIWA. IIWA est un des premiers robots collaboratifs respectant les normes industrielles}
\eng{Designed and realized the development of the Gaia framework in the context of a Renault project. Gaia is a set of reusable components for collaborative IIWA robot platform}

\cveventtechnologies{Java - OSGI - Guice - Git - Maven - Eclipse - Jolokia - AngularJS}

% \item

% \fr{ \cvsubevent{Auditeur}{Infopro}{Septembre 2016}{Sceau - France}}
% \eng{\cvsubevent{Auditor}{Infopro}{September 2016}{Sceau - France}}
% \fr{Aide pour un audit sur l'usine logicielle et la livraison logicielle d'une quinzaine d'applications ayant pour objectif une migration vers une architecture microservices
% }
% \eng{Contributed to the audit of a fifteen applications software factory and of their delivery process, in order to migrate to a microservice architecture}

\fr{\cvsubevent{Développeur / Intégrateur}{ENEDIS - LinkyCom}{Juin 2015 -- Août 2016}{Puteau - France}}
\eng{\cvsubevent{Software Enginner}{ENEDIS - LinkyCom}{June 2015 -- August 2016}{Puteau - France}}
\fr{Modernistation et mise en production du projet LinkyCom est le tuyau de communication entre les 35 millions de compteurs intelligents Linky et les SI ENEDIS.}
\eng{The LinkyCom project is the communication hub between the 35 million Linky smart electric meters and the ENEDIS SI}

\cveventtechnologies{Java - Spring - Jmxtrans - Graphana - Rabbitmq - Ansible - Site Actif/Actif -  Rpm}


\fr{\cvsubevent{Développeur}{BNP CIB - ALMT Tresory}{Septembre 2014 -- Mai 2015}{Paris - France}}
\eng{\cvsubevent{Software Enginner}{BNP CIB - ALMT Tresory}{September 2014 -- May 2015}{Paris - France}}
\fr{Conception et mise en place d'une usine logicielle pour 900 modules de gestion de trésorerie pour les salles de marché, utilisée par 60 développeurs (Inde/France).}
\eng{Designed and implemented a software factory for projects around Kondor (Cash Management Application for trading rooms), used by 60 developers (India / France).}\par
\cveventtechnologies{Bash - Java - Jenkins - Nexus - RPM}

%% If the NEXT page doesn't start with a \cvsection but you'd
%% still like to add a sidebar, then use this command on THIS
%% page to add it. The optional argument lets you pull up the
%% sidebar a bit so that it looks aligned with the top of the
%% main column.
% \addnextpagesidebar[-1ex]{page3sidebar}

\end{document}
