%%%%%%%%%%%%%%%%%
% This is an sample CV template created using altacv.cls
% (v1.1.2, 1 February 2017) written by LianTze Lim (liantze@gmail.com). Now compiles with pdfLaTeX, XeLaTeX and LuaLaTeX.
%
%% It may be distributed and/or modified under the
%% conditions of the LaTeX Project Public License, either version 1.3
%% of this license or (at your option) any later version.
%% The latest version of this license is in
%%    http://www.latex-project.org/lppl.txt
%% and version 1.3 or later is part of all distributions of LaTeX
%% version 2003/12/01 or later.
%%%%%%%%%%%%%%%%

%% If you need to pass whatever options to xcolor
\PassOptionsToPackage{dvipsnames}{xcolor}

%% If you are using \orcid or academicons
%% icons, make sure you have the academicons
%% option here, and compile with XeLaTeX
%% or LuaLaTeX.
% \documentclass[10pt,a4paper,academicons]{altacv}

%% Use the "normalphoto" option if you want a normal photo instead of cropped to a circle
% \documentclass[10pt,a4paper,normalphoto]{altacv}

\documentclass[10pt,a4paper]{altacv}

%% AltaCV uses the fontawesome and academicon fonts
%% and packages.
%% See texdoc.net/pkg/fontawecome and http://texdoc.net/pkg/academicons for full list of symbols.
%%
%% Compile with LuaLaTeX for best results. If you
%% want to use XeLaTeX, you may need to install
%% Academicons.ttf in your operating system's font
%% folder.


% Change the page layout if you need to
\geometry{left=1cm,right=9cm,marginparwidth=6.8cm,marginparsep=1.2cm,top=1.25cm,bottom=1.25cm}

% Change the font if you want to.

% If using pdflatex:
\usepackage[utf8]{inputenc}
\usepackage[T1]{fontenc}
\usepackage[default]{lato}
\usepackage{hyperref}
% If using xelatex or lualatex:
% \setmainfont{Lato}

% Change the colours if you want to
\definecolor{Mulberry}{HTML}{72243D}
\definecolor{SlateGrey}{HTML}{2E2E2E}
\definecolor{LightGrey}{HTML}{666666}
\colorlet{heading}{Sepia}
\colorlet{accent}{Mulberry}
\colorlet{emphasis}{SlateGrey}
\colorlet{body}{LightGrey}

% Change the bullets for itemize and rating marker
% for \cvskill if you want to
\renewcommand{\itemmarker}{{\small\textbullet}}
\renewcommand{\ratingmarker}{\faCircle}


\begin{document}

\fr{\name{Maxence Labusquière}}
\eng{\name{Maxence Labusquiere}}
\tagline{
\fr{Ingénieur développeur}
\eng{Developer engineer}
}


\fr{\photo{3.2cm}{MLabusquiere}}
\personalinfo{%
  % Not all of these are required!
  % You can add your own with \printinfo{symbol}{detail}
  \email{labusquiere@gmail.com}
  \phone{+33 (0)6 67 02 10 42}
  \mailaddress{38 Boulevard de Picpus, 75012 Paris}
  %%Keep this jump
  
  \smallskip
  \twitter{@mlabusquiere}
  \github{github.com/mlabusquiere}
  %% You MUST add the academicons option to \documentclass, then compile with LuaLaTeX or XeLaTeX, if you want to use \orcid or other academicons commands.
%   \orcid{orcid.org/0000-0000-0000-0000}
}

%% Make the header extend all the way to the right, if you want.
\begin{fullwidth}
\makecvheader
\end{fullwidth}

%% Provide the file name containing the sidebar contents as an optional parameter to \cvsection.
%% You can always just use \marginpar{...} if you do
%% not need to align the top of the contents to any
%% \cvsection title in the "main" bar.
\cvsection[mlabusquiere_p1_sidebar]{\fr{Expérience}\eng{Experience}}
\fr{\cvevent{Zenika - Pôle IKI Consulting}{Consultant / Formateur}{Septembre 2014 -- Aujourd'hui}{Paris - France}}
\eng{\cvevent{Zenika - IKI Consulting department}{Consultant / Trainer}{September 2014 -- Today}{Paris - France}}

\eng{As a Zenika consultant during 4 years :}

\begin{itemize}
\fr{
\item Animation d'une trentaine de séances formations
\item Réalisation d'une douzaine de missions d'expertises courtes (RabbitMQ, Java, Jenkins)
\item Contribution aux appels d'offre
\item Participation au recrutement des consultants
\item Animation d'événements pour la tribu Architecture
}
\eng{
\item Animated more than 30 training sessions
\item Realized dozen short expert missions (RabbitMQ, Java, Jenkins)
\item Took part in calls for tender 
\item Contributed to recruiting consultants
\item Animated several internal architecture events
}
\end{itemize}

\divider

\fr{\cvevent{Altice (SFR) - PFS TV - Sekai}{Tech lead / Architecte - Consultant Zenika (Pôle Conseil)}{Mars 2017 -- Aujourd'hui}{Saint Denis - France}}
\eng{\cvevent{Altice (SFR) - PFS TV - Sekai}{Tech lead / Architect - Zenika Consultant (IKI Consulting)}{March 2017 -- Today}{Saint Denis - France}}

\fr{Sekai est l'équipe s'occupant de la refonte des backends de setup des objets IoT SFR et de toutes les applications autour de la TV}
\eng{Sekai is the team that overhaul all backends concerning the tv setup}
\begin{itemize}
\fr{
\item Recueil des besoins, analyse, développement, mise en production et suivi
\item Accompagnement à la définition du socle technique et à son application
\item Gestion des flux d'intégration de certains partenaires via Kafka
\item Mise en place d'une architecture de command sourcing pour gérer les paramétrages de données
\item Développement d'outils de test, d'un framework chassi
}
\eng{
\item Analysed requirements, developed, deployed applications and monitored it
\item Helped the team to define technical standards
\item Defined partners integration flow using kafka
\item Set up a CQRS system
\item Developed test tools and a chassis framework (error handling, logs, ...)
}
\end{itemize}
\printinfo{\faCode}{Java - SpringBoot \fr{écosystème}\eng{ecosystem} - Couchbase - Kafka - Docker - Swarm}
Cucumber

\divider

\fr{\cvevent{Kuka Robotics - Gaia}{Tech lead / Scrum Master - Consultant Zenika (Pôle Conseil)}{Octobre 2016 -- Mars 2017}{Massy-Palaiseau - France}}
\eng{\cvevent{Kuka Robotics - Gaia}{Tech lead / Scrum Master - Zenika Consultant (IKI Consulting)}{October 2016 -- March 2017}{Massy-Palaiseau - France}}
\fr{Gaia est un framework permettant d'adapter sans développeurs les processus métiers du robot IIWA. IIWA est un des premiers robots collaboratifs respectant les normes industrielles}
\eng{Support to the development of the Gaia framework. Gaia is a set of reusable components for their new collaborative IIWA robot and their new Sunrise platform}
\begin{itemize}
\fr{
\item Introduction et mise en place des méthodes agiles (SCRUM)
\item Conception et mise en place de l'architecture du framework Gaia
\item Réalisation d'un projet de marouflage pour Renault
}
\eng{
\item Introduced and implemented agile methods (SCRUM)
\item Designed and realized the Gaia framework
\item Created an application with Gaia framework to allow Renault to paste a part on a car door in motion
}
\end{itemize}
\printinfo{\faCode}{Java - OSGI - Guice - Git - Maven - Eclipse - Jolokia - AngularJS}

\divider


\fr{ \cvevent{Infopro}{Auditeur - Consultant Zenika (Pôle Conseil)}{Septembre 2016}{Sceau - France}}
\eng{\cvevent{Infopro}{Auditor - Zenika Consultant (IKI Consulting)}{September 2016}{Sceau - France}}
\fr{Aide pour un audit sur l'usine logicielle et la livraison logicielle d'une quinzaine d'applications ayant pour objectif une migration vers une architecture microservices
}
\eng{Contributed to the audit of a fifteen applications software factory and of their delivery process, in order to migrate to a microservice architecture}

\clearpage
\cvsection[mlabusquiere_p2_sidebar]{Experience}
%%\divider

\fr{\cvevent{ENEDIS (ERDF) - LinkyCom - Projet Linky}{Développeur / Intégrateur - Consultant Zenika (Pôle Conseil)}{Juin 2015 -- Août 2016}{Puteau - France}}
\eng{\cvevent{ENEDIS (ERDF) - LinkyCom - Linky Project}{Developer - Zenika Consultant (IKI Consulting)}{June 2015 -- August 2016}{Puteau - France}}
\fr{Le projet LinkyCom est le tuyau de communication entre les 35 millions de compteurs intelligents Linky et les SI ENEDIS.}
\eng{The LinkyCom project is the communication hub between the 35 million Linky smart electric meters and the ENEDIS SI}
\begin{itemize}
\fr{
\item Rétro-engineering fonctionnel et refonte de l'architecture du code existant
\item Mise en production et suivie de la production (Support niveau 3)
\item Industrialisation des tests, du déploiement et de la supervision
\item Formation et sensibilisation aux nouveaux processus et outils
}
\eng{
\item Retro-engineered and overhauled existing code / architecture
\item Developed internal testing tools, deployment tests, and supervision architecture
\item Trained and raised awareness on new processes and tools 
}
\end{itemize}
\printinfo{\faCode}{Java 8 - Spring - Jmxtrans - Graphite/Graphana - Rabbitmq - Ansible}
Site Actif/Actif - Git - Maven -  Rpm

\divider

\fr{\cvevent{BNP CIB - Tresory ALMT - IT Core}{Développeur - Consultant Zenika}{Septembre 2014 
-- Mai 2015}{Paris - France}}
\eng{\cvevent{BNP CIB - Tresory ALMT - IT Core}{Developer - Zenika Consultant }{September 2014 -- May 2015}{Paris - France}}
\fr{Conception et mise en place d'une usine logicielle pour les projets autour de Kondor (Application de gestion de trésorerie pour les salles de marché), utilisée par 60 développeurs (Inde/France).}
\eng{Designed and implemented a software factory for projects around Kondor (Cash Management Application for trading rooms), used by 60 developers (India / France).}
\begin{itemize}
\fr{
\item Mise en place de build Mock, d'un outillage développeur et de livraison par RPM, migration de CVS à SVN
\item Conception et mise en oeuvre des processus de releases pour 900 modules (SQL, C, PHP , Java, Ksh, Tibco BusinessWork)
\item Mise en place d'un système de promotion d’artéfacts et de gestion de configuration
\item Formation et sensibilisation sur les nouveaux processus et outils
}
\eng{
\item Designed and developed some tools based on Mock and RPM in the aim to harmonize the release process and the deployment of 900 modules (SQL, C, PHP , Java, Ksh, Tibco BusinessWork)
\item Designed artifact and configuration promotion system 
\item Trained and raised awareness on new processes and tools 
}
\end{itemize}
\printinfo{\faCode}{Bash - Java - Jenkins - Nexus}

\divider

\fr{\cvevent{Devops / Vert.x}{Stagiaire Zenika}{Janvier - Août 2014}{Paris - France}}
\eng{\cvevent{Devops / Vert.x}{Intern - Zenika}{January - August 2014}{Paris - France}}
\begin{itemize}
\fr{
\item Aide à la mise en place de l'offre devops Zenika
\item Création d’une plateforme d'intégration (Microservices) facilitant la mise en place du Continuous Delivery
\item Étude de différents outils du développement (Maven, Gradle, Go ThoughtWork, Jenkins, Chef, Puppet, ect) et des méthodologies de la démarche DevOps
}
\eng{
\item Helped to set up the new Zenika Devops offer
\item Designed a solution and a DSL to facilitate continuous delivery using microservices architecture
\item Analysed different development tools (Maven, Gradle, Go ThoughtWork, Jenkins, Chef, Puppet, ect) and devops methodology approaches
}
\end{itemize}
\printinfo{\faCode}{Vert.x - MongoDB - Hazelcast - Antlr4 - Logstash - Kibana}

\divider

\fr{\cvevent{Fibre Optique / JEE}{Stagiaire Zenika}{Juin - Septembre 2013 }{Paris - France}}
\eng{\cvevent{Optical Fiber / JEE}{Intern - Zenika}{June - Septembre 2013}{Paris - France}}
\fr{Conception et création d’un prototype permettant de gérer les échanges inter opérateurs (définit par l’ARCEP) dans le cadre de la mutualisation de la fibre optique.}
\eng{Prototyped an application to manage inter-operator exchanges (defined by ARCEP) as part of the fiber optics sharing.}
\printinfo{\faCode}{MongoDB - AngularJS -  \fr{écosystème} Spring \eng{stack} - Jenkins - RabbitMQ}

\divider

\fr{\cvevent{Restauration}{Chef de rang}{2008 - 2013 }{France / Angleterre}}
\eng{\cvevent{Restaurants}{Head waiter}{2008 - 2013 }{France / Great Britain}}
\fr{Management d'une équipe de 3 personnes}
\eng{Managed 3 staff members}

%% If the NEXT page doesn't start with a \cvsection but you'd
%% still like to add a sidebar, then use this command on THIS
%% page to add it. The optional argument lets you pull up the
%% sidebar a bit so that it looks aligned with the top of the
%% main column.
% \addnextpagesidebar[-1ex]{page3sidebar}


\end{document}
